\chapter*{Thema und Aufgabenstellung}\thispagestyle{empty}
\label{sec:thema}
\pdfbookmark[0]{Thema und Aufgabenstellung}{sec:thema}

\vspace{-0.2cm}
\textbf{Thema:}\\
Mehrtonsignale zur robusten Distanzschätzung\\

\noindent\textbf{Aufgabenstellung:}\\
\normalsize Im Rahmen der Arbeiten sind Mehrtonsignale zu erzeugen. Für die erzeugten Signale sind Distanzschätzer zu
implementieren (z.B. Phasendifferenz und MUSIC). Zudem sind die Signale bzgl. ihrer Leistungsfähigkeit mit
Hilfe der CRLB zu charakterisieren. Abschließend sollen die Schätzverfahren mit Hilfe einer ebenfalls zu
entwickelnden Kanalsimulationsumgebung zu testen. Optional können die entwickelten Verfahren im realen
Lokalisierungssystem validiert werden. 1. Literaturrecherche: Laufzeitschätzung, inbesondere Mehrtonsignale
2. Theoretische Grenzen: CRLB, Vergleich mit PN-Sequenzen und Zweiton. 3. Erzeugung von
Mehrtonsignalen 4. Entwicklung mehrerer Laufzeitschätzer (z.B. Phasendifferenz und MUSIC) 5.
Kanalsimulation zum Test der entwickelten Verfahren 6. Validierung im realen System (optional)
