\chapter{Einleitung}
\label{chap:Like Vorlage}

Navigation und Ortung spielten seit jeher eine wichtige Rolle für die Menschheit. Vor allem der Welthandel und der damit einhergehenden wirtschaftliche Erfolg einer Nation haben von den Errungenschaften dieser Disziplin profitiert. Handel ist maßgeblich von den logistischen Möglichkeiten eines Landes abhängig, welche damals Schiff- sowie heute Schiff- und Luftfahrt beinhalten. Vor allem in der Schifffahrt hatte es katastrophale Folgen, wenn ein solches Schiff sein Ziel nicht gefunden hat. Deshalb wurden schon im 17. bis 19. Jahrhundert hohe Preisgelder für navigatorische Problemstellungen verliehen, wie für die Lösung des Längengradproblems, welches sich über vier Jahrhunderte zog \cite{Sobel2000}. Mit technischen Fortschritten waren immer bessere Navigationslösungen möglich, bis hin zum \gls{GPS}, oder dem Russischen Pendant \gls{GLONASS}. Mit immer kleiner werdender Mikroelektronik und der heutigen digitalen Vernetzung, bieten sich für diesen Forschungsbereich mehr und mehr Anwendungsmöglichkeiten. Autonomes Fahren, Paketverfolgung oder Robotik sind nur einige Stichworte bezüglich neuer Herausforderungen der Navigation und Ortung. Eine Technologie der letzten Jahre tritt dabei besonders in den Vordergrund: Sensornetze verbinden energieeffiziente miniaturisierte Sensoren in einem intelligenten Netzwerk miteinander. Damit können, räumlich unabhängig, große Mengen an Daten gesammelt und analysiert werden, um z.B. eine globale Wettererfassung zu realisieren. 
Um die neuen Möglichkeiten dieser Technologie zu erforschen, hat die \gls{DFG} die Forschungsgruppe \glslink{BATS}{BATS} zur Entwicklung "`Betriebs-Adaptiver Tracking-Sensorsysteme"' ins Leben gerufen. Diese Systeme sollen die Überwachung einer Vielzahl sich bewegender Objekte ermöglichen. Sie können sehr energieeffizient und klein realisiert werden, wodurch sie ihre Umgebung und die zu beobachtenden Objekte kaum beeinträchtigen. Die Forschungsgruppe, bestehend aus Lehrstühlen der Friedrich-Alexander-Universität Erlangen-Nürnberg, des Leibnitz-Instituts für Evolutions- und Biodiversitätsforschung in Berlin, der Universität Innsbruck und dem \gls{IIS}, will mit diesem System Tierpopulationen beobachten. Dabei soll die Fledermausart "Großes Mausohr" (Myotis myotis) und deren soziales Gefüge genauer untersucht werden. Das Beispiel der Fledermausortung wurde aufgrund der anspruchsvollen Randbedingungen bezüglich Energie, Gewicht, aber auch der mobilen Datenerfassung und Speicherung gewählt. Eine weiter Herausforderung an die Navigationslösung stellt die, für Signalausbreitung ungeeignete Umgebung dar \cite{dfg}. Das System soll in Waldgebieten eingesetzt werden, in welchen sich viele große Objekte befinden, die die elektromagnetischen Wellen bei der Ausbreitung behindern, oder diese reflektieren. Diese Effekte werden Mehrwegeausbreitung und Abschattung genannt. Deren Ursache liegt darin begründet, dass ein Funksignal nicht über die Sichtverbindung, \gls{LOS}, ihr Ziel erreicht, sondern durch ein Objekt abgeschattet, oder an diesem reflektiert wird. Durch Überlagerungen der Wellen am Empfänger, kann es zu erheblichen Abweichungen zum erwarteten Signal kommen. Diese Eigenschaften des Übertragungskanals verursachen fehlerhafte Messergebnisse und führen daher zu Fehlern in der Navigation. Allerdings können mit diesem Wissen auch entsprechende Vorkehrungen getroffen werden, um diese Art von Fehler zu vermeiden. Kern dieser Arbeit ist es, die Eigenschaften eines Mehrwegekanals zu charakterisieren und Verfahren vorzustellen, die zur Vermeidung von Messfehlern angewendet werden können. 

Zu Beginn werden grundlegende Methoden der Funkortung und Navigation erläutert. Anschließend werden die Eigenschaften des Übertragungskanals genauer beleuchtet, um daraufhin geeignete Signale und Schätzalgorithmen zu entwerfen, die Mehrwegefehler minimieren sollen. Es wird untersucht ob Mehrtonsignale einen stark gestörten Übertragungskanal besser auflösen können als Zweitonsignale. Dazu sollen unterschiedliche Methoden, Mehrtonsignale zu erzeugen gezeigt, und deren Eigenschaften erläutert werden. Des weiteren werden Algorithmen zur Distanzschätzung vorgestellt, welche versuchen die Information in solchen Signalen effizient zu nutzen. Insgesamt vier dieser Algorithmen werden in dieser Arbeit untersucht. Zusätzlich werden Möglichkeiten, diese zu testen und zu bewerten auch dargelegt. Abschließend folgt die Implementation und Simulation.   

