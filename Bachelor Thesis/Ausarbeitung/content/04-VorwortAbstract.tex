\clearpage
\thispagestyle{empty}
\chapter*{Kurzzusammenfassung}
\label{sec:Kurzzusammenfassung}
\pdfbookmark[0]{Kurzzusammenfassung}{sec:Kurzzusammenfassung}

\noindent
\emph{Für Navigation und Ortung müssen oftmals präzise Laufzeitmessungen durchgeführt werden. Eine Methode die Laufzeit eines Signals zu ermitteln, ist die Phasendifferenzschätzung. Dafür werden Signale mit mindestens zwei Subträgern benötigt. Es stellte sich heraus, dass ein Zweiton ideal für Phasendifferenzschätzung in einem AWGN-Kanal ist. Er ist jedoch nicht sehr robust gegenüber Mehrwegeausbreitung. In dieser Arbeit sollen Mehrtonsignale mit unterschiedlicher Anzahl und Verteilung von Subträgern zur robusten Distanzschätzung in einer Mehrwegeumgebung untersucht werden. Dazu werden zunächst Methoden zur Erzeugung solcher Signale beschrieben und deren Eigenschaften erklärt. Anschließend werden Schätzverfahren vorgestellt, welche die Parameterschätzung mittels dieser Signale besonders effizient und genau gestalten sollen. Es werden auch Methoden aus der Statistik und Schätztheorie, sowie dem Forschungsgebiet des GPS, vorgestellt, um diese Verfahren zu bewerten. In einer Simulation werden sämtliche Kombinationsmöglichkeiten aus Schätzer und Signal für einen AWGN-Kanal, einen Mehrwegekanal und der Kombination aus beiden ausgewertet. Die Ergebnisse zeigen, dass Mehrtonsignale bis zu einem gewissen Maß Mehrwegeausbreitung auflösen können. Bei besonders schlechten Verhältnissen können jedoch Fehlerwerte unter $\unit[20]{m}$ nicht erreicht werden. Da nicht alle Einflüsse eines Mehrwegekanals berücksichtigt werden konnten, wäre eine Verifizierung in einer realen Umgebung sinnvoll.}



\chapter*{Abstract}
\label{sec:Abstract}
\pdfbookmark[0]{Abstract}{sec:Abstract}

\noindent
\emph{A precise method for estimating the delay of a signal is often needed in navigation and localization. Phase difference of arrival is one method to realize delay estimation. To use this method, a signal with at least two subcarriers is needed. This type of signal is even the optimum for parameter estimation in an AWGN-channel. For distance estimation in multipath propagation channels it is not that well suited though. The purpose of this thesis is to investigate the behavior of mutiton signals with more than two subcarriers in multipath channels. Firstly  signal gerneration methods will be explained, after wich four estimators will be introduced, wich try to use the information, given in the subcarriers, efficiently to increase the estimation accuracy. To have the necessary knowlede to evaluate these estimator an overview of statistical signal processing, estimation theory and an evaluation method from GPS is given. Multiple combinations of the signals and estimators, wich were introduced, will be tested and evaluated in simulation. The results of these simulations show that with multiton signals, estimators have the ability to reduce estimation errors produced by multipath channels. In very bad conditions the estimation error didn't stay below $\unit[20]{m}$. Because the channel modell cant account for all influances of the channel a varification of the simulation results in a real environment would be useful.}
