\chapter{Zusammenfassung und Ausblick}
\label{chap:Schluss}

Im Rahmen dieser Arbeit wurden Mehrtonsignale zur robusten Distanzschätzung untersucht. Dafür mussten Schätzverfahren gefunden und implementiert werden, welche den Informationsgewinn über den Kanal, durch zusätzliche Subträger, ausnutzen können. Robust ist eine Distanzschätzung, wenn selbst bei einem stark gestörten Kanal, wie bei Mehrwegeausbreitung, ein geringer Schätzfehler entsteht. Es wurden fünf Signale mit verschiedenen Subträgerverteilungen und zwei Schätzstrategien, mit jeweils zwei Schätzalgorithmen untersucht. Die erste Strategie, basierend auf Mittelung, beinhaltet einen Schätzer, der Phasendifferenzen zwischen allen benachbarten Subträger bestimmt und anschließend einen Mittelwert bildet. Der zweite Schätzer, welcher dieselbe Strategie verfolgt, wurde von Luise und Reggiannini vorgestellt. Dieser Schätzer verbessert die Genauigkeit der Schätzung, indem er nicht nur Differenzen benachbarter Träger, sondern zwischen allen Trägern bestimmt. Dadurch verkleinert sich jedoch der Eindeutigkeitsbereich der Schätzung. Die Genauigkeit und das \gls{SNR} verbessern sich hingegen. Bei der Verwendung dieses Verfahrens wird eine Vereinfachung getroffen, die nicht auf alle Signalformen zutrifft. Folglich ist dieser Schätzer nicht für alle Signalformen geeignet. Beide Verfahren konnten, mithilfe der zusätzlichen Subträger, Schätzfehler im Mehrwegekanal verringern. Beim Mehrwegekanal ist der Fehler jedoch immer noch zu groß um in realen Systemen eingesetzt zu werden. Der \gls{LuR}-Schätzer musste für die, in dieser Arbeit verwendeten Signale angepasst werden, was seine Leistung minderte. Es ist zu Überlegen, diesen Schätzer mit geeigneten Signalen und einer geeigneten Impulsformung erneut zu Simulieren. 
Die zweite Strategie basiert auf der Zerlegung der Kanalschätzung in einen Signal- und einen Rauschraum. Dadurch, dass Signal- und Rauschanteil getrennt werden, können die zu schätzenden Parameter direkt aus dem Signalraum extrahiert werden. Der MUSIC-Algorithmus erzeugt einen Suchvektor (Steringvektor), welcher alle Verzögerungen durchläuft, und multipliziert ihn mit dem Rauschraum. Entsteht bei der Multiplikation eine Nullstelle, ist die Laufzeit zum gesuchten Vektor gefunden. Dieser Suchprozess ist jedoch rechenaufwendig. Einen anderen Ansatz verfolgt das ESPRIT-Verfahren. Dieses teilt den Signalraum in zwei Frequenzabschnitte, welche durch ein least squares-Problem miteinander verknüpft sind. Zur Bestimmung des gesuchten Parameters muss dieses least squares-Problem gelöst werden.  
Die Methoden, basierend auf einer Zerlegung in Subräume, erreichen zwar nur für einen 7-Ton, im \gls{AWGN}-Kanal die untere Fehlerschranke, können den Mehrwegekanal jedoch sehr gut auflösen. 
Abschließend wurde geschlussfolgert, dass die Subraummethoden in Kombination mit Mehrträgersignalen, bei welchen alle Subträger gleich gewichtet sind, die besten Ergebnisse erzielen. Die Anzahl der Subträger sollte allerdings nicht zu groß gewählt werden, da bei einem kleinem Signal-Rausch-Verhältnis für solche Signale die untere Fehlerschranke nicht erreicht werden kann. Bei der Auswertung wurde mit dem MUSIC-Algorithmus in Kombination einer 7-Träger $m$-Sequenz die besten Ergebnisse erzielt. 


Die Ergebnisse der abschließenden Auswertung sind vielversprechend, da selbst bei einem, im Bezug auf die \gls{CRLB}, nicht idealem Signal, Mehrwegefehler minimiert werden konnten. Der zu beobachtende Schätzfehler ist im Bereich von schlechten Signal-Rausch-Verhältnissen und kurzen Umwegpfaden stark angestiegen. Zudem sollte im Hinterkopf behalten werde, dass das Verhältnis des \gls{SNR} vom \gls{LOS} zu dem des Umwegpfades konstant gehalten wurde. Dies entspricht nicht der Realität, da das \gls{SNR} des Umwegpfades ebenfalls mit zunehmender Distanz abnimmt.  
Es bietet sich an in zukünftigen Arbeiten an der Optimierung der Signalform zu arbeiten, da sich herausgestellt hat, dass unterschiedlich stark gewichtete Subträger, für die in dieser Arbeit untersuchten Schätzer, nicht sehr geeignet sind. Zudem muss auch auf die Impulsformung in zukünftigen Simulationen geachtet werden, da sich herausgestellt hat, dass ein Rechteckimpuls die Gewichte der Subträger verändert. Es wäre denkbar an dieser Stelle eine Sinc-Funktion zur Impulsformung zu verwenden.
Des Weiteren sollte eine Verifizierung der Ergebnisse in einer realen Umgebung stattfinden, da Faktoren die in diesen Betrachtungen vernachlässigt wurden Einflüsse auf die Ergebnisse haben können, wie beispielsweise die zusätzliche Dämpfung des Umwegpfades durch die Reflexion.

