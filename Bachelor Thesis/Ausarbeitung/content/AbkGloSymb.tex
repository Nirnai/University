%%%
%Symbole
%%%
\newglossaryentry{symb:r}{
name=$r$,
description={Distanz die von einer elektromagnetischen Welle zurückgelegt wird},
sort=symbolr ,type=symbolslist
}

\newglossaryentry{symb:tau}{
name=$\tau$,
description={Laufzeit},
sort=symboltau, type=symbolslist
}

\newglossaryentry{symb:c0}{
name=$c_0$,
description={Ausbreitungsgeschwindigkeit einer elektromagnetischen Welle im Vakuum},
sort=symbolc0, type=symbolslist
}

\newglossaryentry{symb:omega}{
name=$\omega$,
description={Kreisfrequenz, entspricht dem überstrichenen Phasenwinkel pro Zeitspanne},
sort=symbolomega , type=symbolslist
}

\newglossaryentry{sym:f}{
name=$f$,
description={Frequenz, Anzahl der Schwingungsperioden pro Zeitspanne},
sort=symbolf , type=symbolslist
}

\newglossaryentry{symb:fc}{
name=$f_c$,
description={Trägerfrequenz der elektromagnetischen Trägerschwingung},
sort=symbolfc , type=symbolslist
}

\newglossaryentry{symb:beta}{
name=$\beta$,
description={Phasenmaß},
sort=symbolbeta , type=symbolslist
}

\newglossaryentry{symb:epsilon}{
name=$\epsilon$,
description={absolute Dielektrizitätszahl},
sort=symbolepsilon , type=symbolslist
}

\newglossaryentry{symb:mu}{
name=$\mu$,
description={absolute Permeabilitätszahl},
sort=symbolmu , type=symbolslist
}

\newglossaryentry{symb:dphi}{
name=$\Delta \varphi$,
description={Phasendifferenz zwischen Subträgerfrequenzen},
sort=symboldphi , type=symbolslist
}

\newglossaryentry{symb:df}{
name=$\Delta f$,
description={Frequenzabstand zweier Subträgerfrequenzen},
sort=symboldf , type=symbolslist
}

\newglossaryentry{symb:sigman}{
name=$\sigma_n$,
description={Standardabweichung des Rauschprozesses},
sort=symbolsigman , type=symbolslist
}

\newglossaryentry{symb:B}{
name=$B$,
description={Bandbreite},
sort=symbolB, type=symbolslist
}

\newglossaryentry{symb:N0}{
name=$N_0$,
description={Rauschleistung für eine gegebene Bandbreite},
sort=symbolN0 ,type=symbolslist
}

\newglossaryentry{symb:C}{
name=$C$,
description={Signalleistung},
sort=symbolC ,type=symbolslist
}

\newglossaryentry{symb:PTx}{
name=$P_{Tx}$,
description={Sendesignalleistung},
sort=symbolPTx ,type=symbolslist
}

\newglossaryentry{symb:brms}{
name=$\beta_{rms}$,
description={Effektive Bandbreite die von einem Signal benötigt wird. Ist signalformabhängig},
sort=symbolbrms ,type=symbolslist
}

\newglossaryentry{symb:phi}{
name=$\varphi$,
description={Phasendrehung},
sort=symbolphi ,type=symbolslist
}

\newglossaryentry{symb:}{
name=,
description={},
sort=symbol ,type=symbolslist
}



%\newglossaryentry{symb:Pi}{
%name=$\pi$,
%description={Die Kreiszahl.},
%sort=symbolpi, type=symbolslist
%}
%\newglossaryentry{symb:Phi}{
%name=$\varphi$,
%description={Ein beliebiger Winkel.},
%sort=symbolphi, type=symbolslist
%}
%\newglossaryentry{symb:Lambda}{
%name=$\lambda$,
%description={Eine beliebige Zahl, mit der der nachfolgende Ausdruck
%multipliziert wird.},
%sort=symbollambda, type=symbolslist
%}

%%%
%Abk�rzungen
%%%
\newacronym{GPS}{GPS}{Global Positioning System}
\newacronym{GLONASS}{GLONASS}{Global Navigation Satellite System}
\newacronym{DFG}{DFG}{Deutsche
Forschungsgemeinschaft}
\newacronym{BATS}{BATS}{Betriebs-Adaptive Tracking-Sensorsysteme}
\newacronym{IIS}{IIS}{Fraunhofer-Institut für Integrierte Schaltungen}
\newacronym{LOS}{LOS}{line of sight}
\newacronym{AoA}{AoA}{angle of arrival}
\newacronym{AWGN}{AWGN}{Additiv White Gaussian Noise}
\newacronym{SNR}{SNR}{signal-to-noise ratio}
\newacronym{WDF}{WDF}{Wahrscheinlichkeitsdichtefunktion}
\newacronym{mse}{mse}{mean square error}
\newacronym{mvue}{mvue}{minimum variance unbiased-estimator}
\newacronym{CRLB}{CRLB}{Cramer-Rao lower bound}
\newacronym{DFT}{DFT}{Diskrete Fourier-Transformation}
\newacronym{LuR}{L$\&$R}{Luise und Reggiannini}
\newacronym{rmse}{rmse}{root mean square error}
\newacronym{rms}{rms}{root mean square}
\newacronym{MUSIC}{MUSIC}{Multiple Signal Classification}
\newacronym{ESPRIT}{ESPRIT}{Estimation of Signal Parameters via Rotational Invarience Techniques}
\newacronym{FD}{FD-Filter}{Fractional-Delay-Filter}


%\newacronym{CD}{CD}{Compact Disc}
%Eine Abk�rzung mit Glossareintrag
%\newacronym{AD}{AD}{Active Directory\protect\glsadd{glos:AD}}


%%%
%Glossareintr�ge
%%%
%\newglossaryentry{glos:AD}{
%name=Active Directory,
%description={Active Directory ist in einem Windows 2000/" "Windows
%Server 2003-Netzwerk der Verzeichnisdienst, der die zentrale
%Organisation und Verwaltung aller Netzwerkressourcen erlaubt. Es
%erm�glicht den Benutzern �ber eine einzige zentrale Anmeldung den
%Zugriff auf alle Ressourcen und den Administratoren die zentral
%organisierte Verwaltung, transparent von der Netzwerktopologie und
%den eingesetzten Netzwerkprotokollen. Das daf�r ben�tigte
%Betriebssystem ist entweder Windows 2000 Server oder
%Windows Server 2003, welches auf dem zentralen
%Dom�nencontroller installiert wird. Dieser h�lt alle Daten des
%Active Directory vor, wie z.B. Benutzernamen und
%Kennw�rter.}
%}
%\newglossaryentry{glos:AntwD}{name=Antwortdatei, description={Informationen zum
%Installieren einer Anwendung oder des Betriebssystems.}}
