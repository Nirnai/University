% ~~~~~~~~~~~~~~~~~~~~~~~~~~~~~~~~~~~~~~~~~~~~~~~~~~~~~~~~~~~~~~~~~~~~~~~~
% Fonts Fonts Fonts
% ~~~~~~~~~~~~~~~~~~~~~~~~~~~~~~~~~~~~~~~~~~~~~~~~~~~~~~~~~~~~~~~~~~~~~~~~

% Alle Schriften die hier angegeben sind sehen im PDF richtig aus.
% Die LaTeX Standardschrift ist die Latin Modern (lmodern Paket).
% If Latin Modern is not available for your distribution you must install the
% package cm-super instead. Otherwise your fonts will look horrible in the PDF

% DO NOT LOAD ae Package for the font !

%% ==== Zusammengesetzte Schriften  (Sans + Serif) =======================

%% - Latin Modern
\usepackage{lmodern}
%% -------------------
%
%% - Times, Helvetica, Courier (Word Standard...)
%\usepackage{mathptmx}
%\usepackage[scaled=.90]{helvet}
%\usepackage{courier}
%% -------------------
%%
%% - Palantino , Helvetica, Courier
%\usepackage{mathpazo}
%\usepackage[scaled=.95]{helvet}
%\usepackage{courier}
%% -------------------
%
%% - Bera Schriften
%\usepackage{bera}
%% -------------------
%
%% - Charter, Bera Sans
%\usepackage{charter}\linespread{1.05}
%\renewcommand{\sfdefault}{fvs}

%% ===== Serifen =========================================================

%\usepackage{mathpazo}                 %% --- Palantino
%\usepackage{charter}\linespread{1.05} %% --- Charter
%\usepackage{bookman}                  %% --- Bookman (laedt Avant Garde !!)
%\usepackage{newcent}                  %% --- New Century Schoolbook (laedt Avant Garde !!)

%\usepackage[%                         %% --- Fourier
%   upright,     % Math fonts are upright
%   expert,      % Only for EXPERT Fonts!
%   oldstyle,    % Only for EXPERT Fonts!
%   fulloldstyle % Only for EXPERT Fonts!
%]{fourier} %



%% ===== Sans Serif ======================================================

%\usepackage[scaled=.95]{helvet}        %% --- Helvetica
%\usepackage{cmbright}                  %% --- CM-Bright (eigntlich eine Familie)
%\usepackage{tpslifonts}                %% --- tpslifonts % Font for Slides
%\usepackage{avantgar}                  %% --- Avantgarde

%%%% =========== Italics ================

%\usepackage{chancery}                  %% --- Zapf Chancery

%%%% =========== Typewriter =============

%\usepackage{courier}                   %% --- Courier
%\renewcommand{\ttdefault}{cmtl}        %% --- CmBright Typewriter Font
%\usepackage[%                          %% --- Luxi Mono (Typewriter)
%   scaled=0.9
%]{luximono}



%%%% =========== Mathe ================

%% Recommanded to use with fonts: Aldus, Garamond, Melior, Sabon
%\usepackage[                           %% --- EulerVM (MATH)
%   small,       %for smaller Fonts
%  euler-digits % digits in euler fonts style
%]{eulervm}

% \usepackage[
% %   utopia,
% %   garamond,
%    charter
% ]{mathdesign}

%%%% (((( !!! kommerzielle Schriften !!! )))))))))))))))))))))))))))))))))))))))))))))))))))

%% ===== Serifen (kommerzielle Schriften ) ================================

%% --- Adobe Aldus
%\renewcommand{\rmdefault}{pasx}
%\renewcommand{\rmdefault}{pasj} %%oldstyle digits
% math recommended: \usepackage[small]{eulervm}

%% --- Adobe Garamond
%\usepackage[%
%   osf,        % oldstyle digits
%   scaled=1.05 %appropriate in many cases
%]{xagaramon}
% math recommended: \usepackage{eulervm}

%% --- Adobe Stempel Garamond
%\renewcommand{\rmdefault}{pegx}
%\renewcommand{\rmdefault}{pegj} %%oldstyle digits

%% --- Adobe Melior
%\renewcommand{\rmdefault}{pml}
% math recommended: %\usepackage{eulervm}

%% --- Adobe Minion
%\renewcommand{\rmdefault}{pmnx}
%\renewcommand{\rmdefault}{pmnj} %oldstyle digits
% math recommended: \usepackage[small]{eulervm} or \usepackage{mathpmnt} % commercial

%% --- Adobe Sabon
%\renewcommand{\rmdefault}{psbx}
%\renewcommand{\rmdefault}{psbj} %oldstyle digits
% math recommended: \usepackage{eulervm}

%% --- Adobe Times
% math recommended: \usepackage{mathptmx} % load first !
%\renewcommand{\rmdefault}{ptmx}
%\renewcommand{\rmdefault}{ptmj} %oldstyle digits

%% --- Linotype ITC Charter
%\renewcommand{\rmdefault}{lch}

%% --- Linotype Meridien
%\renewcommand{\rmdefault}{lmd}

%%% ===== Sans Serif (kommerzielle Schriften) ============================

%% --- Adobe Frutiger
%\usepackage[
%   scaled=0.90
%]{frutiger}

%% --- Adobe Futura (=Linotype FuturaLT) : Sans Serif
%\usepackage[
%   scaled=0.94  % appropriate in many cases
%]{futura}

%% --- Adobe Gill Sans : Sans Serif
%\usepackage{gillsans}

%% -- Adobe Myriad  : Sans Serif
%\renewcommand{\sfdefault}{pmy}
%\renewcommand{\sfdefault}{pmyc} %% condensed Font

%% --- Syntax : sans serif font
%\usepackage[
%   scaled
%]{asyntax}

%% --- Adobe Optima : Semi Sans Serif
%\usepackage[
%   medium %darker medium weight fonts
%]{optima}

%% --- Linotype ITC Officina Sans
%\renewcommand{\sfdefault}{lo9}



