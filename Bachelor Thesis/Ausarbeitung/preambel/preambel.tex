%%% Frei nach einer Vorlage von Matthias Pospiech
%%% Modifiziert von Frederik Beer



%%% Packages for LaTeX -programming
% Define commands that don't eat spaces.
\usepackage{xspace}
% IfThenElse
\usepackage{ifthen}
%%% Doc: ftp://tug.ctan.org/pub/tex-archive/macros/latex/contrib/oberdiek/ifpdf.sty
% command for testing for pdf-creation
\usepackage{ifpdf} %\ifpdf \else \fi

%%% Internal Commands: ----------------------------------------------
\makeatletter
%
\providecommand{\IfPackageLoaded}[2]{\@ifpackageloaded{#1}{#2}{}}
\providecommand{\IfPackageNotLoaded}[2]{\@ifpackageloaded{#1}{}{#2}}
\providecommand{\IfElsePackageLoaded}[3]{\@ifpackageloaded{#1}{#2}{#3}}
%
\newboolean{chapteravailable}%
\setboolean{chapteravailable}{false}%

\ifcsname chapter\endcsname
  \setboolean{chapteravailable}{true}%
\else
  \setboolean{chapteravailable}{false}%
\fi


\providecommand{\IfChapterDefined}[1]{\ifthenelse{\boolean{chapteravailable}}{#1}{}}%
\providecommand{\IfElseChapterDefined}[2]{\ifthenelse{\boolean{chapteravailable}}{#1}{#2}}%

\providecommand{\IfDefined}[2]{%
\ifcsname #1\endcsname
   #2 %
\else
     % do nothing
\fi
}
%
% Check for 'draft' mode - commands.
\newcommand{\IfNotDraft}[1]{\ifx\@draft\@undefined #1 \fi}
\newcommand{\IfNotDraftElse}[2]{\ifx\@draft\@undefined #1 \else #2 \fi}
\newcommand{\IfDraft}[1]{\ifx\@draft\@undefined \else #1 \fi}
%



% Definde frontmatter, mainmatter and backmatter if not defined
\@ifundefined{frontmatter}{%
   \newcommand{\frontmatter}{%
      %In Roemischen Buchstaben nummerieren (i, ii, iii)
      \pagenumbering{roman}
   }
}{}
\@ifundefined{mainmatter}{%
   % scrpage2 benoetigt den folgenden switch
   % wenn \mainmatter definiert ist.
   \newif\if@mainmatter\@mainmattertrue
   \newcommand{\mainmatter}{%
      % -- Seitennummerierung auf Arabische Zahlen zuruecksetzen (1,2,3)
      \pagenumbering{arabic}%
      \setcounter{page}{1}%
   }
}{}
\@ifundefined{backmatter}{%
   \newcommand{\backmatter}{
      %In Roemischen Buchstaben nummerieren (i, ii, iii)
      \pagenumbering{roman}
   }
}{}

% Pakete speichern die spaeter geladen werden sollen
\newcommand{\LoadPackagesNow}{}
\newcommand{\LoadPackageLater}[1]{%
   \g@addto@macro{\LoadPackagesNow}{%
      \usepackage{#1}%
   }%
}



\makeatother
%%% ----------------------------------------------------------------
%
%%% Doc: www.cs.brown.edu/system/software/latex/doc/calc.pdf
\usepackage{calc}

%%% Doc: ftp://tug.ctan.org/pub/tex-archive/macros/latex/contrib/xcolor/xcolor.pdf
\usepackage[
	table % Load for using rowcolors command in tables
]{xcolor}


%%% Doc: http://www.ctan.org/tex-archive/macros/latex/contrib/listings/listings.pdf
\usepackage{listings}

%% Settings for the listing stuff
\definecolor{hellgrau}{rgb}{0.9,0.9,0.9}
\definecolor{colKeys}{rgb}{0,0,1}
\definecolor{colIdentifier}{rgb}{0,0,0}
\definecolor{colComments}{rgb}{1,0,0}
\definecolor{colString}{rgb}{0,0.5,0}

\lstset{%
   morekeywords={AND,ASC,avg,CHECK,COMMIT,count,DECODE,DESC,DISTINCT,%
                 GROUP,IN,LIKE,NUMBER,ROLLBACK,SUBSTR,sum,VARCHAR2}%
}
\lstset{%
    float=hbp,%
    basicstyle=\ttfamily\small, %
    %identifierstyle=\color{colIdentifier}, %
    %keywordstyle=\color{colKeys}, %
    %stringstyle=\color{colString}, %
    %commentstyle=\color{colComments}, %
    columns=flexible, %
    tabsize=2, %
    frame=single, %
    extendedchars=true, %
    showspaces=false, %
    showstringspaces=false, %
    numbers=left, %
    numberstyle=\tiny, %
    breaklines=true, %
    backgroundcolor=\color{hellgrau}, %
    breakautoindent=true, %
    captionpos=b%
}

%%% Doc: ftp://tug.ctan.org/pub/tex-archive/macros/latex/required/babel/babel.pdf
\usepackage[
%	german,
	ngerman,
%	english,
%	frensh,
]{babel}


%%% Doc: ftp://tug.ctan.org/pub/tex-archive/macros/latex/required/graphics/grfguide.pdf
% Will be loaded by pstools
\usepackage[pdftex]{graphicx}

%%% Doc: http://mirrors.ctan.org/macros/latex/contrib/pstool/pstool.pdf
\usepackage{pstool}

%%% Doc: ftp://tug.ctan.org/pub/tex-archive/macros/latex/contrib/oberdiek/epstopdf.pdf
%%% Macht nur Sinn bei der Verwendung von PDFLatex und EPS Grafiken, wandelt diese dann
%%% automatisch in PDFs um.
\usepackage{epstopdf}


%%% Doc: ftp://tug.ctan.org/pub/tex-archive/macros/latex/required/amslatex/math/amsldoc.pdf
\usepackage[
   centertags, % (default) center tags vertically
   %tbtags,    % 'Top-or-bottom tags': For a split equation, place equation numbers level
               % with the last (resp. first) line, if numbers are on the right (resp. left).
   sumlimits,  %(default) Place the subscripts and superscripts of summation
               % symbols above and below
   %nosumlimits, % Always place the subscripts and superscripts of summation-type
               % symbols to the side, even in displayed equations.
   intlimits,  % Like sumlimits, but for integral symbols.
   %nointlimits, % (default) Opposite of intlimits.
   namelimits, % (default) Like sumlimits, but for certain 'operator names' such as
               % det, inf, lim, max, min, that traditionally have subscripts placed underneath
               % when they occur in a displayed equation.
   %nonamelimits, % Opposite of namelimits.
   %leqno,     % Place equation numbers on the left.
   %reqno,     % Place equation numbers on the right.
   %fleqn,     % Position equations at a fixed indent from the left margin rather than
               % centered in the text column.
]{amsmath} %
\usepackage{amssymb}
\usepackage{trfsigns}
%% Doc: ftp://tug.ctan.org/pub/tex-archive/macros/latex/contrib/marginnote/marginnote.pdf
\usepackage{marginnote}

%% Doc: (inside relsize.sty )
%% ftp://tug.ctan.org/pub/tex-archive/macros/latex/contrib/misc/relsize.sty
\usepackage{relsize}

%% Doc: ftp://tug.ctan.org/pub/tex-archive/macros/latex/contrib/ms/ragged2e.pdf
\usepackage{ragged2e}

\usepackage[T1]{fontenc} % T1 Schrift Encoding
\usepackage{textcomp}	 % Zusatzliche Symbole (Text Companion font extension)

%%% Schriften werden in Fonts.tex geladen
% ~~~~~~~~~~~~~~~~~~~~~~~~~~~~~~~~~~~~~~~~~~~~~~~~~~~~~~~~~~~~~~~~~~~~~~~~
% Fonts Fonts Fonts
% ~~~~~~~~~~~~~~~~~~~~~~~~~~~~~~~~~~~~~~~~~~~~~~~~~~~~~~~~~~~~~~~~~~~~~~~~

% Alle Schriften die hier angegeben sind sehen im PDF richtig aus.
% Die LaTeX Standardschrift ist die Latin Modern (lmodern Paket).
% If Latin Modern is not available for your distribution you must install the
% package cm-super instead. Otherwise your fonts will look horrible in the PDF

% DO NOT LOAD ae Package for the font !

%% ==== Zusammengesetzte Schriften  (Sans + Serif) =======================

%% - Latin Modern
\usepackage{lmodern}
%% -------------------
%
%% - Times, Helvetica, Courier (Word Standard...)
%\usepackage{mathptmx}
%\usepackage[scaled=.90]{helvet}
%\usepackage{courier}
%% -------------------
%%
%% - Palantino , Helvetica, Courier
%\usepackage{mathpazo}
%\usepackage[scaled=.95]{helvet}
%\usepackage{courier}
%% -------------------
%
%% - Bera Schriften
%\usepackage{bera}
%% -------------------
%
%% - Charter, Bera Sans
%\usepackage{charter}\linespread{1.05}
%\renewcommand{\sfdefault}{fvs}

%% ===== Serifen =========================================================

%\usepackage{mathpazo}                 %% --- Palantino
%\usepackage{charter}\linespread{1.05} %% --- Charter
%\usepackage{bookman}                  %% --- Bookman (laedt Avant Garde !!)
%\usepackage{newcent}                  %% --- New Century Schoolbook (laedt Avant Garde !!)

%\usepackage[%                         %% --- Fourier
%   upright,     % Math fonts are upright
%   expert,      % Only for EXPERT Fonts!
%   oldstyle,    % Only for EXPERT Fonts!
%   fulloldstyle % Only for EXPERT Fonts!
%]{fourier} %



%% ===== Sans Serif ======================================================

%\usepackage[scaled=.95]{helvet}        %% --- Helvetica
%\usepackage{cmbright}                  %% --- CM-Bright (eigntlich eine Familie)
%\usepackage{tpslifonts}                %% --- tpslifonts % Font for Slides
%\usepackage{avantgar}                  %% --- Avantgarde

%%%% =========== Italics ================

%\usepackage{chancery}                  %% --- Zapf Chancery

%%%% =========== Typewriter =============

%\usepackage{courier}                   %% --- Courier
%\renewcommand{\ttdefault}{cmtl}        %% --- CmBright Typewriter Font
%\usepackage[%                          %% --- Luxi Mono (Typewriter)
%   scaled=0.9
%]{luximono}



%%%% =========== Mathe ================

%% Recommanded to use with fonts: Aldus, Garamond, Melior, Sabon
%\usepackage[                           %% --- EulerVM (MATH)
%   small,       %for smaller Fonts
%  euler-digits % digits in euler fonts style
%]{eulervm}

% \usepackage[
% %   utopia,
% %   garamond,
%    charter
% ]{mathdesign}

%%%% (((( !!! kommerzielle Schriften !!! )))))))))))))))))))))))))))))))))))))))))))))))))))

%% ===== Serifen (kommerzielle Schriften ) ================================

%% --- Adobe Aldus
%\renewcommand{\rmdefault}{pasx}
%\renewcommand{\rmdefault}{pasj} %%oldstyle digits
% math recommended: \usepackage[small]{eulervm}

%% --- Adobe Garamond
%\usepackage[%
%   osf,        % oldstyle digits
%   scaled=1.05 %appropriate in many cases
%]{xagaramon}
% math recommended: \usepackage{eulervm}

%% --- Adobe Stempel Garamond
%\renewcommand{\rmdefault}{pegx}
%\renewcommand{\rmdefault}{pegj} %%oldstyle digits

%% --- Adobe Melior
%\renewcommand{\rmdefault}{pml}
% math recommended: %\usepackage{eulervm}

%% --- Adobe Minion
%\renewcommand{\rmdefault}{pmnx}
%\renewcommand{\rmdefault}{pmnj} %oldstyle digits
% math recommended: \usepackage[small]{eulervm} or \usepackage{mathpmnt} % commercial

%% --- Adobe Sabon
%\renewcommand{\rmdefault}{psbx}
%\renewcommand{\rmdefault}{psbj} %oldstyle digits
% math recommended: \usepackage{eulervm}

%% --- Adobe Times
% math recommended: \usepackage{mathptmx} % load first !
%\renewcommand{\rmdefault}{ptmx}
%\renewcommand{\rmdefault}{ptmj} %oldstyle digits

%% --- Linotype ITC Charter
%\renewcommand{\rmdefault}{lch}

%% --- Linotype Meridien
%\renewcommand{\rmdefault}{lmd}

%%% ===== Sans Serif (kommerzielle Schriften) ============================

%% --- Adobe Frutiger
%\usepackage[
%   scaled=0.90
%]{frutiger}

%% --- Adobe Futura (=Linotype FuturaLT) : Sans Serif
%\usepackage[
%   scaled=0.94  % appropriate in many cases
%]{futura}

%% --- Adobe Gill Sans : Sans Serif
%\usepackage{gillsans}

%% -- Adobe Myriad  : Sans Serif
%\renewcommand{\sfdefault}{pmy}
%\renewcommand{\sfdefault}{pmyc} %% condensed Font

%% --- Syntax : sans serif font
%\usepackage[
%   scaled
%]{asyntax}

%% --- Adobe Optima : Semi Sans Serif
%\usepackage[
%   medium %darker medium weight fonts
%]{optima}

%% --- Linotype ITC Officina Sans
%\renewcommand{\sfdefault}{lo9}





%%% Doc: ftp://tug.ctan.org/pub/tex-archive/macros/latex/contrib/mh/doc/mathtools.pdf
\usepackage[fixamsmath,disallowspaces]{mathtools}

%%% Doc: http://www.ctan.org/info?id=fixmath
\usepackage{fixmath}

%%% Doc: ftp://tug.ctan.org/pub/tex-archive/macros/latex/contrib/onlyamsmath/onlyamsmath.dvi
\usepackage[
	all,
	warning
]{onlyamsmath}

%%% Doc: http://www.tex.ac.uk/ctan/macros/latex/contrib/koma-script/tocstyle.pdf
%%% Kann verwendet werden, wenn im Inhaltsverzeichnis �berall oder nirgends Punkte gew�nscht sind.
%\usepackage{tocstyle}
%\usetocstyle{allwithdot}
%\usetocstyle{noonewithdot}
%\usetocstyle{KOMAlike}

%------------------------------------------------------

% -- Vektor fett darstellen -----------------
% \let\oldvec\vec
% \def\vec#1{{\boldsymbol{#1}}} %Fetter Vektor
% \newcommand{\ve}{\vec} %
% -------------------------------------------

%%% Doc: ftp://tug.ctan.org/pub/tex-archive/macros/latex/contrib/was/icomma.dtx
\usepackage{icomma}

%%% Tauschen von Epsilon und andere:
% \let\ORGvarrho=\varrho
% \let\varrho=\rho
% \let\rho=\ORGvarrho
%
\let\ORGvarepsilon=\varepsilon
\let\varepsilon=\epsilon
\let\epsilon=\ORGvarepsilon
%
% \let\ORGvartheta=\vartheta
% \let\vartheta=\theta
% \let\theta=\ORGvartheta
%
% \let\ORGvarphi=\varphi
% \let\varphi=\phi
% \let\phi=\ORGvarphi


%%% Doc: ftp://tug.ctan.org/pub/tex-archive/macros/latex/contrib/booktabs/booktabs.pdf
\usepackage{booktabs}

% Tabellen ueber mehere Seiten
% ----------------------------
%%% Doc: ftp://tug.ctan.org/pub/tex-archive/macros/latex/contrib/carlisle/ltxtable.pdf
% \usepackage{ltxtable} % Longtable + tabularx
                        % (multi-page tables) + (auto-sized columns in a fixed width table)
% -> nach hyperref laden
\LoadPackageLater{ltxtable}


%%% Doc: ftp://tug.ctan.org/pub/tex-archive/macros/latex/contrib/soul/soul.pdf
\usepackage{soul}		            % Unterstreichen, Sperren
%%% Doc: ftp://tug.ctan.org/pub/tex-archive/macros/latex/contrib/misc/url.sty
\usepackage{url} % Setzen von URLs. In Verbindung mit hyperref sind diese auch aktive Links.

%%%% Doc: ftp://tug.ctan.org/pub/tex-archive/macros/latex/contrib/footmisc/footmisc.pdf
\usepackage[
   bottom,      % Footnotes appear always on bottom. This is necessary
                % especially when floats are used
   stable,      % Make footnotes stable in section titles
   perpage,     % Reset on each page
   %para,       % Place footnotes side by side of in one paragraph.
   %side,       % Place footnotes in the margin
   ragged,      % Use RaggedRight
   %norule,     % suppress rule above footnotes
   multiple,    % rearrange multiple footnotes intelligent in the text.
   %symbol,     % use symbols instead of numbers
]{footmisc}

%% Einruecken der Fussnote einstellen
%\setlength\footnotemargin{10pt}

%--- footnote counter documentweit durchlaufend ------------------------------
%\usepackage{chngcntr}
%\counterwithout{footnote}{chapter}
%-----------------------------------------------------------------------------


%%% Doc: Documentation inside dtx File
\usepackage[ngerman]{varioref} % Intelligente Querverweise


%%% Doc: ftp://tug.ctan.org/pub/tex-archive/macros/latex/contrib/enumitem/enumitem.pdf
% Better than 'paralist' and 'enumerate' because it uses a keyvalue interface !
% Do not load together with enumerate.
\IfPackageNotLoaded{enumerate}{
	\usepackage{enumitem}
}

%% Doc: ftp://tug.ctan.org/pub/tex-archive/macros/latex/contrib/csquotes/csquotes.pdf
% Advanced features for clever quotations
\usepackage[%
   babel,            % the style of all quotation marks will be adapted
                     % to the document language as chosen by 'babel'
   german=quotes,		% Styles of quotes in each language
   english=british,
   french=guillemets
]{csquotes}

% All facilities which take a 'cite' argument will not insert
% it directly. They pass it to an auxiliary command called \mkcitation
% which  may be redefined to format the citation.
\renewcommand*{\mkcitation}[1]{{\,}#1}
\renewcommand*{\mkccitation}[1]{ #1}

\SetBlockThreshold{2} % Anzahl von Zeilen

\newenvironment{myquote}%
	{\begin{quote}\small}%
	{\end{quote}}%
\SetBlockEnvironment{myquote}
%\SetCiteCommand{} % Changes citation command


%%% Doc: ftp://tug.ctan.org/pub/tex-archive/macros/latex/contrib/natbib/natbib.pdf
\usepackage[%
	%round,	%(default) for round parentheses;
	square,	% for square brackets;
	%curly,	% for curly braces;
	%angle,	% for angle brackets;
	%colon,	% (default) to separate multiple citations with colons;
	comma,	% to use commas as separaters;
	%authoryear,% (default) for author-year citations;
	numbers,	% for numerical citations;
	%super,	% for superscripted numerical citations, as in Nature;
	sort,		% orders multiple citations into the sequence in which they appear in the list of references;
	sort&compress,    % as sort but in addition multiple numerical citations
                   % are compressed if possible (as 3-6, 15);
	%longnamesfirst,  % makes the first citation of any reference the equivalent of
                   % the starred variant (full author list) and subsequent citations
                   %normal (abbreviated list);
	%sectionbib,      % redefines \thebibliography to issue \section* instead of \chapter*;
                   % valid only for classes with a \chapter command;
                   % to be used with the chapterbib package;
	%nonamebreak,     % keeps all the authors names in a citation on one line;
                   %causes overfull hboxes but helps with some hyperref problems.
]{natbib}

%%% Bibliography styles according to DIN
%%% get from: http://www.ctan.org/tex-archive/biblio/bibtex/contrib/german/din1505/
%\bibliographystyle{alphadin}
%\bibliographystyle{abbrvdin}
\bibliographystyle{bib/bst/plaindin}
%\bibliographystyle{unsrtdin}
%\bibliographystyle{bib/bst/alphadin-mod} % Modifiziert: Kleinere Abstaende vor ";" und kein "+" bei etal.

%%% Bibliography styles created with custombib
%%% Doc: ftp://tug.ctan.org/pub/tex-archive/macros/latex/contrib/custom-bib/makebst.pdf
%\bibliographystyle{bib/bst/AlphaDINFirstName}
%\bibliographystyle{bib/bst/alphadin}


% weitere BibTeX styles: http://www.cs.stir.ac.uk/~kjt/software/latex/showbst.html

%%% Doc: ftp://tug.ctan.org/pub/tex-archive/macros/latex/contrib/microtype/microtype.pdf
\ifpdf
\usepackage[%
	expansion=true, % better typography, but with much larger PDF file.
	protrusion=true
]{microtype}
\fi

%%% Doc: ftp://tug.ctan.org/pub/tex-archive/macros/latex/contrib/hyperref/doc/manual.pdf

\usepackage[
	  % Farben fuer die Links
    colorlinks=false,         % Links erhalten Farben statt Kaeten
    urlcolor=pdfurlcolor,    % \href{...}{...} external (URL)
    filecolor=pdffilecolor,  % \href{...} local file
    linkcolor=pdflinkcolor,  %\ref{...} and \pageref{...}
    % Links
    raiselinks=true,			 % calculate real height of the link
    %breaklinks,              % Links berstehen Zeilenumbruch, geht nicht bei DVI->PS
    backref=page,            % Backlinks im Literaturverzeichnis (section, slide, page, none)
    pagebackref=true,        % Backlinks im Literaturverzeichnis mit Seitenangabe
    verbose,
    hyperindex=true,         % backlinkex index
    linktocpage=true,        % Inhaltsverzeichnis verlinkt Seiten
    hyperfootnotes=false,     % Keine Links auf Fussnoten
    % Bookmarks
    bookmarks=true,          % Erzeugung von Bookmarks fuer PDF-Viewer
    bookmarksopenlevel=1,    % Gliederungstiefe der Bookmarks
    bookmarksopen=true,      % Expandierte Untermenues in Bookmarks
    bookmarksnumbered=false,  % Nummerierung der Bookmarks
    %bookmarkstype=toc,       % Art der Verzeichnisses
    % Anchors
    plainpages=false,        % Anchors even on plain pages ?
    pageanchor=true,         % Pages are linkable
    % PDF Informationen
    pdftitle={},             % Titel
    pdfauthor={Autor},            % Autor
    pdfcreator={LaTeX, hyperref, KOMA-Script}, % Ersteller
    %pdfproducer={pdfeTeX 1.10b-2.1} %Produzent
    pdfstartview=FitH,       % Dokument wird Fit Width geaefnet
    pdfpagemode=UseOutlines, % Bookmarks im Viewer anzeigen
    %pdfpagelabels=true,      % set PDF page labels
		%dvipdfm,								 % Links auch wenn PDF �ber DVI Umweg erstellt wird
		pdfborder={0 0 0},
 ]{hyperref}


\IfPackageLoaded{backref}{
   % % Change Layout of Backref
   \renewcommand*{\backref}[1]{%
   	% default interface
   	% #1: backref list
   	%
   	% We want to use the alternative interface,
   	% therefore the definition is empty here.
   }%
   \renewcommand*{\backrefalt}[4]{%
   	% alternative interface
   	% #1: number of distinct back references
   	% #2: backref list with distinct entries
   	% #3: number of back references including duplicates
   	% #4: backref list including duplicates
   	\mbox{(Zitiert auf %
   	\ifnum#1=1 %
		   Seite~%
	   \else
   		Seiten~%
   	\fi
   	#2)}%
   }
}

%%% Doc: ftp://tug.ctan.org/pub/tex-archive/macros/latex/contrib/oberdiek/hypcap.pdf
% Links auf Gleitumgebungen springen nicht zur Beschriftung,
% sondern zum Anfang der Gleitumgebung
\IfPackageLoaded{hyperref}{%
	\usepackage[figure]{hypcap}
}

% Auch Abbildung und nicht nur die Nummer wird zum Link (abgeleitet
% aus Posting von Heiko Oberdiek (d09n5p$9md$1@news.BelWue.DE);
% Verwendung: In \abbvref{label} ist ein Beispiel dargestellt
\providecommand*{\abbvrefname}{Abbildung}
\newcommand*{\abbvref}[1]{%
  \hyperref[#1]{\abbvrefname}\vref{#1}%
}

%%% Doc: ftp://tug.ctan.org/pub/tex-archive/macros/latex/contrib/pdfpages/pdfpages.pdf
\usepackage{pdfpages} % Include pages from external PDF documents in LaTeX documents

% Pakete Laden die nach Hyperref geladen werden sollen
\LoadPackagesNow % (ltxtable, tabularx)


%%% Doc: only dtx Package
\usepackage{float}             % Stellt die Option [H] fuer Floats zur Verfgung

%%% Doc: No Documentation
\usepackage{flafter}          % Floats immer erst nach der Referenz setzen

% Defines a \FloatBarrier command, beyond which floats may not
% pass; useful, for example, to ensure all floats for a section
% appear before the next \section command.
%\usepackage[
%	section		% "\section" command will be redefined with "\FloatBarrier"
%]{placeins}


%%% Doc: ftp://tug.ctan.org/pub/tex-archive/macros/latex/contrib/subfig/subfig.pdf
\usepackage{subfig} % Layout wird weiter unten festgelegt !

%%% Doc: ftp://tug.ctan.org/pub/tex-archive/macros/latex/contrib/wrapfig/wrapfig.sty
\usepackage{wrapfig}	        % defines wrapfigure and wrapfloat
%\setlength{\wrapoverhang}{\marginparwidth} % aeerlapp des Bildes ...
%\addtolength{\wrapoverhang}{\marginparsep} % ... in den margin
\setlength{\intextsep}{0.75\baselineskip} % Platz ober- und unterhalb des Bildes
% \intextsep ignoiert bei draft ???
%\setlength{\columnsep}{1em} % Abstand zum Text


% Make float placement easier
\renewcommand{\floatpagefraction}{.75} % vorher: .5
\renewcommand{\textfraction}{.1}       % vorher: .2
\renewcommand{\topfraction}{.8}        % vorher: .7
\renewcommand{\bottomfraction}{.5}     % vorher: .3
\setcounter{topnumber}{3}              % vorher: 2
\setcounter{bottomnumber}{2}           % vorher: 1
\setcounter{totalnumber}{5}            % vorher: 3

%%% Doc: http://tug.ctan.org/tex-archive/macros/latex/contrib/auto-pst-pdf/auto-pst-pdf.pdf
%\usepackage[
%latex={-interaction=nonstopmode},
%crop=off,runs=2
%]{auto-pst-pdf} %use [off] to stop compilation

%%% Doc: ftp://tug.ctan.org/pub/tex-archive/macros/latex/contrib/psfrag/pfgguide.pdf
%\usepackage{psfrag}	% Ersetzen von Zeichen in eps Bildern

%%% Doc: http://www.ctan.org/tex-archive/macros/latex/contrib/sidecap/sidecap.pdf
\usepackage[%
%	outercaption,%	(default) caption is placed always on the outside side
%	innercaption,% caption placed on the inner side
%	leftcaption,%  caption placed on the left side
	rightcaption,% caption placed on the right side
%	wide,%			caption of float my extend into the margin if necessary
%	margincaption,% caption set into margin
	ragged,% caption is set ragged
]{sidecap}

\renewcommand\sidecaptionsep{2em}
%\renewcommand\sidecaptionrelwidth{20}
\sidecaptionvpos{table}{c}
\sidecaptionvpos{figure}{c}

%%% Seems to be needed for glossaries on newer miktex installations
\usepackage{datatool}
%%% Doc: http://mirror.informatik.uni-mannheim.de/pub/mirrors/tex-archive/macros/latex/contrib/glossaries/glossaries-manual.html
\usepackage[ngerman]{translator}
%Paket laden
\usepackage[
nonumberlist, %keine Seitenzahlen anzeigen
acronym,      %ein Abk�rzungsverzeichnis erstellen
toc,          %Eintr�ge im Inhaltsverzeichnis
section]      %im Inhaltsverzeichnis auf section-Ebene erscheinen
{glossaries}

%Ein eigenes Symbolverzeichnis erstellen
\newglossary[slg]{symbolslist}{syi}{syg}{Symbolverzeichnis}

%Den Punkt am Ende jeder Beschreibung deaktivieren
\renewcommand*{\glspostdescription}{}

%Glossar-Befehle anschalten
\makeglossaries


%%% Doc: http://tug.ctan.org/tex-archive/macros/latex/contrib/siunitx/siunitx.pdf
\usepackage[]{units}
\usepackage{siunitx}

%Normale im LaTeX-Dokument verwendete Schriftart nutzen
\sisetup{detect-all}

%%% Doc: http://ftp.gwdg.de/pub/ctan/macros/latex/contrib/ellipsis/ellipsis.pdf
\usepackage{ellipsis}  % >>Intelligente<< \dots


%%% Doc: ftp://tug.ctan.org/pub/tex-archive/macros/latex/contrib/setspace/setspace.sty
\usepackage{setspace}
%\doublespace	        % 2-facher Abstand
\onehalfspace        % 1,5-facher Abstand



% BCOR
%    current  % Satzspiegelberechnung mit dem aktuell gültigen BCOR-Wert erneut
%             % durchführen.
% DIV
%    calc     % Satzspiegelberechnung einschließlich Ermittlung eines guten
%             % DIV-Wertes erneut durchführen.
%    classic  % Satzspiegelberechnung nach dem
%             % mittelalterlichen Buchseitenkanon
%             % (Kreisberechnung) erneut durchführen.
%    current  % Satzspiegelberechnung mit dem aktuell gültigen DIV-Wert erneut
%             % durchführen.
%    default  % Satzspiegelberechnung mit dem Standardwert für das aktuelle
%             % Seitenformat und die aktuelle Schriftgröße erneut durchführen.
%             % Falls kein Standardwert existiert calc anwenden.
%    last     % Satzspiegelberechnung mit demselben DIV -Argument, das beim
%             % letzten Aufruf angegeben wurde, erneut durchführen

%\raggedbottom     % Variable Seitenhoehen zulassen

% Farben ================================================================

\IfDefined{definecolor}{%

% Farbe der Ueberschriften
%\definecolor{sectioncolor}{RGB}{0, 51, 153} % Blau
%\definecolor{sectioncolor}{RGB}{0, 25, 152}    % Blau (dunkler))
\definecolor{sectioncolor}{RGB}{0, 0, 0}    % Schwarz
%
% Farbe des Textes
\definecolor{textcolor}{RGB}{0, 0, 0}        % Schwarz
%
% Farbe fuer grau hinterlegte Boxen (fuer Paket framed.sty)
\definecolor{shadecolor}{gray}{0.90}

% Farben fuer die Links im PDF
\definecolor{pdfurlcolor}{rgb}{0.6,0,0}
\definecolor{pdffilecolor}{rgb}{0,0.5,0}
\definecolor{pdflinkcolor}{rgb}{0,0,0.75}

% Farben fuer Listings
\colorlet{stringcolor}{green!40!black!100}
\colorlet{commencolor}{blue!0!black!100}

} % Endif

%% Aussehen der URLS======================================================

%fuer URL (nur wenn url geladen ist)
\IfDefined{urlstyle}{
	\urlstyle{tt} %sf
}

%% Kopf und Fusszeilen====================================================
%%% Doc: ftp://tug.ctan.org/pub/tex-archive/macros/latex/contrib/koma-script/scrguide.pdf
\usepackage[%
   automark,         % automatische Aktualisierung der Kolumnentitel
   nouppercase,      % Grossbuchstaben verhindern
   %markuppercase    % Grossbuchstaben erzwingen
   %markusedcase     % vordefinierten Stil beibehalten
   %komastyle,       % Stil von Koma Script
   %standardstyle,   % Stil der Standardklassen
]{scrpage2}

\IfElseChapterDefined{%
   \pagestyle{scrheadings} % Seite mit Headern
}{
   \pagestyle{scrplain} % Seiten ohne Header
}
%\pagestyle{empty} % Seiten ohne Header
\clearscrheadings
%\clearscrplain
%
% Was steht wo...
\IfElseChapterDefined{
   % Oben aussen: Kapitel und Section
   % Unten aussen: Seitenzahl
   % \ohead{\headmark} % Oben außen: Setzt Kapitel und Section automatisch
   % \ofoot[\pagemark]{\pagemark}
   % oder...
   % Oben aussen: Seitenzahlen
   % Oben innen: Kapitel und Section
   \cfoot{\pagemark}
   \ohead{\headmark}
}{
   \cfoot[\pagemark]{\pagemark} % Mitte unten: Seitenzahlen bei plain
}
% Vollstaendige Liste der moeglichen Positionierungen
% \lehead[scrplain-links-gerade]{scrheadings-links-gerade}
% \cehead[scrplain-mittig-gerade]{scrheadings-mittig-gerade}
% \rehead[scrplain-rechts-gerade]{scrheadings-rechts-gerade}
% \lefoot[scrplain-links-gerade]{scrheadings-links-gerade}
% \cefoot[scrplain-mittig-gerade]{scrheadings-mittig-gerade}
% \refoot[scrplain-rechts-gerade]{scrheadings-rechts-gerade}
% \lohead[scrplain-links-ungerade]{scrheadings-links-ungerade}
% \cohead[scrplain-mittig-ungerade]{scrheadings-mittig-ungerade}
% \rohead[scrplain-rechts-ungerade]{scrheadings-rechts-ungerade}
% \lofoot[scrplain-links-ungerade]{scrheadings-links-ungerade}
% \cofoot[scrplain-mittig-ungerade]{scrheadings-mittig-ungerade}
% \rofoot[scrplain-rechts-ungerade]{scrheadings-rechts-ungerade}
% \ihead[scrplain-innen]{scrheadings-innen}
% \chead[scrplain-zentriert]{scrheadings-zentriert}
% \ohead[scrplain-außen]{scrheadings-außen}
% \ifoot[scrplain-innen]{scrheadings-innen}
% \cfoot[scrplain-zentriert]{scrheadings-zentriert}
% \ofoot[scrplain-außen]{scrheadings-außen}


%\usepackage{lastpage} % Stellt 'LastPage' zur Verfuegung
%\cfoot[Seite \pagemark~von \pageref{LastPage}]{} % Seitenzahl von Anzahl Seiten

% Angezeigte Abschnitte im Header
\IfElseChapterDefined{
   \automark[section]{chapter} %[rechts]{links}
}{
   \automark[subsection]{section} %[rechts]{links}
}
%
% Linien (moegliche Kombination mit Breiten)
\IfChapterDefined{
   %\setheadtopline{}     % modifiziert die Parameter fuer die Linie ueber dem Seitenkopf
   \setheadsepline{.4pt}[\color{black}]
                         % modifiziert die Parameter fuer die Linie zwischen Kopf
                         % und Textkörper
   %\setfootsepline{}    % modifiziert die Parameter fuer die Linie zwischen Text
                         % und Fuß
   %\setfootbotline{}    % modifiziert die Parameter fuer die Linie unter dem Seitenfuss
}

% Groesse des Headers
\setlength{\headheight}{1.1\baselineskip}
% -> eingestellt �ber Option 'headlines'.

% Breite von Kopf und Fusszeile einstellen
% \setheadwidth[Verschiebung]{Breite}
% \setfootwidth[Verschiebung]{Breite}
% m�gliche Werte
% paper - die Breite des Papiers
% page - die Breite der Seite
% text - die Breite des Textbereichs
% textwithmarginpar - die Breite des Textbereichs inklusive dem Seitenrand
% head - die aktuelle Breite des Seitenkopfes
% foot - die aktuelle Breite des Seitenfusses
\setheadwidth[0pt]{text}
\setfootwidth[0pt]{text}


%% Fussnoten =============================================================
% Keine hochgestellten Ziffern in der Fussnote (KOMA-Script-spezifisch):
\deffootnote{1.5em}{1em}{\makebox[1.5em][l]{\thefootnotemark}}
\addtolength{\skip\footins}{\baselineskip} % Abstand Text <-> Fussnote

\setlength{\dimen\footins}{10\baselineskip} % Beschraenkt den Platz von Fussnoten auf 10 Zeilen

\interfootnotelinepenalty=10000 % Verhindert das Fortsetzen von
                                % Fussnoten auf der gegenüberligenden Seite


%% Schriften (Sections )==================================================

\IfElsePackageLoaded{fourier}{
   \newcommand\SectionFontStyle{\rmfamily}
}{
   \newcommand\SectionFontStyle{\sffamily}
}

% -- Koma Schriften --
\IfChapterDefined{%
   \setkomafont{chapter}{\huge\SectionFontStyle}    % Chapter
}
\setkomafont{sectioning}{\SectionFontStyle} %  % Titelzeilen % \bfseries
\setkomafont{pagenumber}{\small\SectionFontStyle}             % Seitenzahl
\setkomafont{pageheadfoot}{\small\sffamily}        % Kopfzeile
%\setkomafont{pagefoot}{\small\sffamily}        % Kopfzeile
\setkomafont{descriptionlabel}{\itshape}        % Kopfzeile
%

\addtokomafont{sectioning}{\color{sectioncolor}} % Farbe der Ueberschriften
\IfChapterDefined{%
	\addtokomafont{chapter}{\color{sectioncolor}} % Farbe der Ueberschriften
}
\renewcommand*{\raggedsection}{\raggedright} % Titelzeile linksbuendig, haengend
%
%% UeberSchriften (Chapter und Sections) =================================
% -- Ueberschriften komlett Umdefinieren --
%%% Doc: ftp://tug.ctan.org/pub/tex-archive/macros/latex/contrib/titlesec/titlesec.pdf
\usepackage{titlesec}

% -- Section Aussehen veraendern --
% --------------------------------
%% -> Section mit Unterstrich
% \titleformat{\section}
%   [hang]%[frame]display
%   {\usekomafont{sectioning}\Large}
%  {\thesection}
%   {6pt}
%   {}
%   [\titlerule \vspace{0.5\baselineskip}]
% --------------------------------

% -- Chapter Aussehen veraendern --
% --------------------------------
%--> Box mit (Kapitel + Nummer ) +  Name
% \titleformat{\chapter}[display]     % {command}[shape]
%   {\usekomafont{chapter}\filcenter} % format
%   {                                 % label
%   {\fcolorbox{black}{shadecolor}{
%   {\huge\chaptertitlename\mbox{\hspace{1mm}}\thechapter}
%   }}}
%   {1pc}                             % sep (from chapternumber)
%   {\vspace{1pc}}                    % {before}[after] (before chaptertitle and after)
% --------------------------------
%--> Kapitel + Nummer + Trennlinie + Name + Trennlinie
\titleformat{\chapter}[display]	% {command}[shape]
  {\usekomafont{chapter}\Large \color{black}}	% format
  {   										% label
  \LARGE\MakeUppercase{\chaptertitlename} \Huge \thechapter \filright%
  }%}
  {1pt}										% sep (from chapternumber)
  {\titlerule \vspace{0.9pc} \filright \color{sectioncolor}}   % {before}[after] (before chaptertitle and after)
  [\color{black} \vspace{0.9pc} \filright {\titlerule}]


%% Captions (Schrift, Aussehen) ==========================================

% % Folgende Befehle werden durch das Paket caption und subfig ersetzt !
% \setcapindent{1em} % Einrueckung der Beschriftung
% \setkomafont{caption}{\color{black}\small\sffamily\RaggedRight}  % Schrift fuer Caption
% \setkomafont{captionlabel}{\color{black}\small}   % Schrift fuer 'Abbildung' usw.

%%% Doc: ftp://tug.ctan.org/pub/tex-archive/macros/latex/contrib/caption/caption.pdf
\usepackage{caption}
% Aussehen der Captions
\captionsetup{
   margin = 10pt,
   font = {small,sf},
   labelfont = {small,bf},
   format = plain, % oder 'hang'
   indention = 0em,  % Einruecken der Beschriftung
   labelsep = colon, %period, space, quad, newline
   justification = RaggedRight, % justified, centering
   singlelinecheck = true, % false (true=bei einer Zeile immer zentrieren)
   position = bottom %top
}
%%% Bugfix Workaround
\DeclareCaptionOption{parskip}[]{}
\DeclareCaptionOption{parindent}[]{}

% Aussehen der Captions fuer subfigures (subfig-Paket)
\IfPackageLoaded{subfig}{
 \captionsetup[subfloat]{%
   margin = 10pt,
   font = {small,sf},
   labelfont = {small,bf},
   format = plain, % oder 'hang'
   indention = 0em,  % Einruecken der Beschriftung
   labelsep = space, %period, space, quad, newline
   justification = RaggedRight, % justified, centering
   singlelinecheck = true, % false (true=bei einer Zeile immer zentrieren)
   position = bottom, %top
   labelformat = parens % simple, empty % Wie die Bezeichnung gesetzt wird
 }
}

% Aendern der Bezeichnung fuer Abbildung und Tabelle
% \addto\captionsngerman{% "captionsgerman" fuer alte  Rechschreibung
%   \renewcommand{\figurename}{Abb.}%
%   \renewcommand{\tablename}{Tab.}%
% }

% Caption fuer nicht fliessende Umgebungen
%%% Doc: ftp://tug.ctan.org/pub/tex-archive/macros/latex/contrib/misc/capt-of.sty
\IfPackageNotLoaded{caption}{
	\usepackage{capt-of} % only load when caption is not loaded. Otherwise compiling will fail.
	%Usage: \captionof{table}[short Titel]{long Titel}
}
%


%%% Doc: ftp://tug.ctan.org/pub/tex-archive/macros/latex/contrib/mcaption/mcaption.pdf
% Captions in Margins
% \usepackage[
% 	top,
% 	bottom
% ]{mcaption}

%%% Example:
% \begin{figure}
%   \begin{margincap}[short caption]{margin caption}
%     \centering
%     \includegraphics{picture}
%   \end{margincap}
% \end{figure}



% \numberwithin{figure}{chapter} %Befehl zum Kapitelweise Nummerieren der Bilder, setzt `amsmath' vorraus
% \numberwithin{table}{chapter}  %Befehl zum Kapitelweise Nummerieren der Tabellen, setzt `amsmath' vorraus

%% Inhaltsverzeichnis (Schrift, Aussehen) sowie weitere Verzeichnisse ====

\setcounter{secnumdepth}{4}    % Abbildungsnummerierung mit groesserer Tiefe
\setcounter{tocdepth}{2}		 % Inhaltsverzeichnis mit groesserer Tiefe
%

% Inhalte von List of Figures
\IfPackageLoaded{subfig}{
	\setcounter{lofdepth}{1}  %1 = nur figures, 2 = figures + subfigures
}


% Auszufuehrende Befehle  ------------------------------------------------
\IfDefined{makeindex}{\makeindex}
\IfDefined{makenomenclature}{\makenomenclature}
\IfPackageLoaded{minitoc}{\ifundefined{chapter}{\dosecttoc}{\dominitoc}}


\listfiles
%------------------------------------------------------------------------

