\begin{frame}{Control and Navigation}

\begin{itemize}
\item
  Bestimmen der Pose aus Odometry

  \begin{itemize}
  \item
    x und y direkt aus odom und theta aus quaternion
  \item
    Erweiterung mit gmapping tfs map and base\_footprint um drift freie
    lokalisierung zu bekommen
  \end{itemize}
\item
  Bekommt ein Ziel als x und y coordinate und bestimmt nötige rotation
  selbständig

  \begin{itemize}
  \tightlist
  \item
    Entscheidung in welche Richtung gedreht werden soll
  \end{itemize}
\item
  P-Regler für die letzte Meter, um Genauigkeit zu erhöhen

  \begin{itemize}
  \tightlist
  \item
    Eventuell noch auf PID erweitern
  \end{itemize}
\item
  Erkennt Objekte aus Lidar und stoppt wenn zu nahe

  \begin{itemize}
  \tightlist
  \item
    Noch zu machen: avoidance path
  \end{itemize}
\end{itemize}

\end{frame}

\begin{frame}{Object Detection}

\begin{itemize}
\item
  Sucht kontinuierlich im LIDAR nach Objekten und published deren Posen.
\item
  Führt die Initialisierung durch:

  \begin{itemize}
  \item
    Es wird nach 3er Objektreihen gesucht, die 2 Bedingungen erfüllen:

    \begin{itemize}
    \item
      Das Verhältnis zwischen den 2 inneren Abstanden beträgt ca. 33\%
    \item
      Ale 3 Objekte liegen auf einer Geraden.
    \end{itemize}
  \item
    Berechnung von a/b mittels Kosinussatz
  \item
    Berechnung von der eigenen Position im Spielfeld, Spielfeld und
    gegnerischem Tor
  \end{itemize}
\item
  Initiale Messung wird etwa 10x durchgeführt
\item
  Je nach Abweichung der Ergebnisse werden sie akzeptiert oder nicht.
\item
  Wenn nicht, alterative Initialisierung durch Bewegung.
\end{itemize}

\end{frame}

\begin{frame}{Image Processing}

\begin{itemize}
\item
  Läuft auf dem Turtlebot
\item
  Sucht nach den blau/gelben Pucks und den grünen Pfosten

  \begin{itemize}
  \item
    Thresholding der Farben im Bild in HSV
  \item
    Canny Kantendetektor über Threshold Img
  \item
    Konturen Suche mit findContours
  \item
    Berechnung der HuMoments für Konturen ab einer gewissen Größe
  \item
    Bestimmung des Objekts anhand Farbe und der 7 HuMoments
  \item
    Bestimmung des Mittelpunktes der Objekte im Bild
  \item
    Bestimmung der 3D Pose des Objekts aus der Punktwolke
  \end{itemize}
\item
  Bestimmt Teamfarbe

  \begin{itemize}
  \item
    Nach der Initialisierung fahren wir seitlich vor das Tor und nehmen
    ein Bild auf.
  \item
    Im Bild wird nach einer gelben Kontur gesucht und deren Fläche wird
    approximiert.
  \end{itemize}
\end{itemize}

\end{frame}

\begin{frame}{Angelina}

\begin{itemize}
\item
  Umwandlung der testgui als in ausführbaren node
\item
  ergänzen eines topics zur Kommunikation mit dem turtlebot
\end{itemize}

\end{frame}
